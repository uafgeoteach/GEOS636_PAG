\documentclass[unknownkeysallowed]{beamer}
\mode<presentation>
{
%  \usetheme{AnnArbor}
%  \usetheme{Dresden}
%  \usetheme{Montpellier}
%  \usetheme{Antibes}
%  \usetheme{Frankfurt}
%  \usetheme{PaloAlto}
%  \usetheme{Bergen}
%  \usetheme{Boadilla}
%  \usetheme{Goettingen}
%  \usetheme{Pittsburgh}	%!!
%  \usetheme{Berkeley}
%  \usetheme{Hannover}
%  \usetheme{Rochester}		%!!!
%  \usetheme{Berlin}
%  \usetheme{Ilmenau}
%  \usetheme{Singapore}
  \usetheme{Boadilla}		%viel platz
%  \usetheme{JuanLesPins}
%  \usetheme{Szeged}		%!
%  \usetheme{boxes}
%  \usetheme{Luebeck}
%  \usetheme{Warsaw}
%  \usetheme{Copenhagen}
%  \usetheme{Madrid}
%  \usetheme{Darmstadt}
%  \usetheme{Malmoe}
%  \usetheme{default}
%  \usetheme{JuanLesPins}

%  \usetheme{Marburg}


%\usefonttheme{professionalfonts}
%	default | professionalfonts | serif |
%	structurebold | structureitalicserif |
%	structuresmallcapsserif
%\useinnertheme{rounded}
%	circles | default | inmargin |
%	rectangles | rounded

%  \setbeamercovered{transparent}
  % oder auch nicht
\usecolortheme{rose}


\definecolor{uaf yellow}{cmyk}{0,0.16,1,0} % official UAF yellow
\definecolor{light yellow}{cmyk}{0.01,0,0.16,0}
\definecolor{uaf blue}{cmyk}{1,0.66,0,0.02} % official UAF blue
\definecolor{light blue}{cmyk}{0.22,0.11,0,0}
\definecolor{arsc blue}{HTML}{005496}
\definecolor{arsc red}{HTML}{a20a42}
\definecolor{arsc green}{HTML}{009a82}
\definecolor{light gray}{HTML}{777777}

  %navigation aus, klaut nur platz
  \setbeamertemplate{navigation symbols}{}
% Reset title background to default
%\setbeamertemplate{title page}[default]
\setbeamercolor{title}{bg=}
\setbeamercolor{frametitle}{bg=uaf blue, fg=white}
\setbeamercolor{institute}{fg=white}
\setbeamercolor{date}{fg=white}
\setbeamercolor{block}{bg=}
%\setbeamercolor{title}{fg=black}

% Reset block background to default
%\setbeamertemplate{blocks}[default]
%\setbeamercolor{block title}{bg=}
%\setbeamercolor{block body}{bg=}

\beamertemplatenavigationsymbolsempty  
\setbeamertemplate{blocks}[rounded][shadows=false]

\useinnertheme{circles}

}
\usepackage[latin1]{inputenc}
\usepackage{latexsym}
\usepackage{amsfonts}
%\usepackage{natbib}
\usepackage{fancyhdr}
\usepackage{graphicx}
%\usepackage{subfigure}
% oder was auch immer
\usepackage{grffile}
\usepackage{pgf}
\usepackage{tikz}

\usepackage{listings}

\usepackage{times}
\usepackage[T1]{fontenc}
%\usepackage{appendixnumber}
% Oder was auch immer. Zu beachten ist, das Font und Encoding passen
% m�ssen. Falls T1 nicht funktioniert, kann man versuchen, die Zeile
% mit fontenc zu l�schen.

\hypersetup{
    bookmarks=true,         % show bookmarks bar?
    unicode=false,          % non-Latin characters in Acrobat's bookmarks
    pdftoolbar=true,        % show Acrobat's toolbar?
    pdfmenubar=true,        % show Acrobat's menu?
    pdffitwindow=false,     % window fit to page when opened
    pdfstartview={FitH},    % fits the width of the page to the window
    pdftitle={My title},    % title
    pdfauthor={Author},     % author
    pdfsubject={Subject},   % subject of the document
    pdfcreator={Creator},   % creator of the document
    pdfproducer={Producer}, % producer of the document
    pdfkeywords={keyword1} {key2} {key3}, % list of keywords
    pdfnewwindow=true,      % links in new window
    colorlinks=false,       % false: boxed links; true: colored links
    linkcolor=red,          % color of internal links
    citecolor=green,        % color of links to bibliography
    filecolor=magenta,      % color of file links
    urlcolor=cyan           % color of external links
}

\title[PAG]% (optional, nur bei langen Titeln n�tig)
{GEOS 436 / 636\\
Programming and Automation for Geoscientists\\[20pt]
-- Week 04: Functions --
}

\author[Grapenthin]% (optional, nur bei vielen Autoren)
{Ronni Grapenthin\\
rgrapenthin@alaska.edu\\
Elvey 413B\\
x7682}
% - Namen m�ssen in derselben Reihenfolge wie im Papier erscheinen.
% - Der \inst{?} Befehl sollte nur verwendet werden, wenn die Autoren
%   unterschiedlichen Instituten angeh�ren.

\institute[UAF] % (optional, aber oft n�tig)
{}
% - Der \inst{?} Befehl sollte nur verwendet werden, wenn die Autoren
%   unterschiedlichen Instituten angeh�ren.
% - Keep it simple, niemand interessiert sich f�r die genau Adresse.

% - Namen m�ssen in derselben Reihenfolge wie im Papier erscheinen.
% - Der \inst{?} Befehl sollte nur verwendet werden, wenn die Autoren
%   unterschiedlichen Instituten angeh�ren.

% - Der \inst{?} Befehl sollte nur verwendet werden, wenn die Autoren
%   unterschiedlichen Instituten angeh�ren.
% - Keep it simple, niemand interessiert sich f�r die genau Adresse.

\date[September 16, 2020] % (optional, sollte der abgek�rzte Konferenzname sein)
{September 16, 2020}

% - Volle oder abgek�rzter Name sind m�glich.
% - Dieser Eintrag ist nicht f�r das Publikum gedacht (das wei�
%   n�mlich, bei welcher Konferenz es ist), sondern f�r Leute, die diese
%   Folien sp�ter lesen.

%\AtBeginSection[]
%{
%  \begin{frame}<beamer>
%    \frametitle{Outline}
%    \tableofcontents[currentsection,currentsubsection]
%  \end{frame}
%}

% Falls Aufz�hlungen immer schrittweise gezeigt werden sollen, kann
% folgendes Kommando benutzt werden:

%\beamerdefaultoverlayspecification{<+->}

%%switch on to have only frame numbers
\setbeamertemplate{footline}[frame number]

\defbeamertemplate*{title page}{customized}[1][]
{
		\begin{tikzpicture}
			\node[text width=\textwidth,
				fill=gray!70, 
				fill opacity=0.75,
				text opacity=1,
				rounded corners = 10pt,
				inner sep=2pt]{
				\begin{center}	
			  \usebeamerfont{title}{\bf \usebeamercolor[fg]{title} \inserttitle}
			  \par
			  \usebeamerfont{subtitle}\insertsubtitle\par
			  \bigskip
			  \usebeamerfont{author}\insertauthor\par
			  \bigskip
			  \usebeamerfont{institute}\insertinstitute\par
			  \bigskip
			  \usebeamerfont{date}\insertdate\par
			  \end{center}
			  };
	\end{tikzpicture}		  
%	\vspace{0.4cm}\usebeamercolor[fg]{titlegraphic}\inserttitlegraphic 
%	\begin{flushright}
%	\vspace{-1.25cm}\includegraphics[width=2cm]{../moore_logo_transp.png}\vspace{5cm}
%	\end{flushright}
}

\begin{document}

\lstset{numbers=left, numberstyle=\tiny, stepnumber=2, basicstyle=\ttfamily, numbersep=5pt, xleftmargin=10pt}

\setbeamertemplate{background}{\includegraphics[width=\paperwidth]{/home/roon/Pictures/rooftop_initial.jpg}}

	\begin{frame}
	\begin{center}
		\titlepage
	\end{center}
	\end{frame}

\setbeamertemplate{background}{}

\begin{frame}
\frametitle{}
	\vspace{2cm}
	\begin{center}
		Wouldn't it be great if we could think up some functionality, write it up,
		test it, and be able to reuse it at any time in the future?$^{*}$
	\end{center}
	\vspace{4cm}
	{\tiny {\color{gray}$^{*}$hint: yes!}}
\end{frame}

\begin{frame}
	\frametitle{Function}
		\vspace{-0.5cm}
			\begin{center}
				\includegraphics[width=0.75\textwidth]{/home/roon/work/teaching/2017/computing_tools/gfx/wikipedia_function1.png}
			\end{center}
			\begin{flushright}
				\tiny{\emph{\url{https://en.wikipedia.org}}}
			\end{flushright}
\end{frame}

\begin{frame}
	\frametitle{Function \& Function}
			\begin{center}
				\includegraphics[width=0.5\textwidth]{/home/roon/work/teaching/2017/computing_tools/gfx/wikipedia_function2.png}
			\end{center}
			\begin{flushright}
				\tiny{\emph{\url{https://en.wikipedia.org}}}
			\end{flushright}
\end{frame}

\begin{frame}
	\frametitle{Use a Function in a Function}
			\vspace{-0.25cm}
			\begin{center}
				\includegraphics[width=0.5\textwidth]{/home/roon/work/teaching/2017/computing_tools/gfx/wikipedia_function3.png}
			\end{center}
			\begin{flushright}
				\tiny{\emph{after: \url{https://en.wikipedia.org}}}
			\end{flushright}
\end{frame}

\begin{frame}
\frametitle{What Is A Function?}
	\begin{center}
		\dots a set of rules that takes input, from which it produces output.
	\end{center}
\end{frame}

\begin{frame}
\frametitle{What Is A Function?}
	\begin{itemize}
		\item one of the boxes you drew in the flow charts
		\item one of the most important tools you have to:
			\begin{itemize}
				\item capture \& contain
				\item test
				\item reuse
			\end{itemize}
			algorithms / behavior of any complexity 
		\item change something in {\bf one} spot instead of all over your source code (don't copy and paste in programming!)
	\end{itemize}
\end{frame}

\begin{frame}
\frametitle{Examples}
	\vspace{-0.5cm}
	\begin{columns}[t]
		\column{0.48\textwidth}
		\begin{block}{}
		{\scriptsize
		 \lstinputlisting[language=python]{../listings/grades.py}
		}
		\end{block}
		\begin{block}{}
		{\scriptsize
		 \lstinputlisting[language=python, numbers=none]{../listings/grades_output_script.txt}
		}
		\end{block}
		\column{0.48\textwidth}
		\uncover<2->{
			\begin{block}{}
			{\scriptsize
			 \lstinputlisting[language=bash]{../listings/letter_grades.bash}
			}
			\end{block}
			\begin{block}{}
			{\scriptsize
			 \lstinputlisting[language=bash, numbers=none]{../listings/grades_shell_output_script.txt}
			}
			\end{block}
		}
	\end{columns}
\end{frame}

\begin{frame}
\frametitle{Examples - Interactive Python}
		\begin{block}{import {\bf entire module}}
		{\scriptsize
		 \lstinputlisting[language=python, numbers=none]{../listings/grades_output.txt}
		}
		\end{block}
		\uncover<2->{
			\begin{block}{import {\bf individual function} from a module}
			{\scriptsize
			 \lstinputlisting[language=python, numbers=none]{../listings/grades_output2.txt}
			}
			\end{block}
		}
\end{frame}

\begin{frame}
\frametitle{Examples - bash vs tcsh}
	\vspace{-0.5cm}
	\begin{columns}[t]
		\column{0.48\textwidth}
		\begin{block}{}
		{\scriptsize
		 \lstinputlisting[language=bash, emph={letter_grade},emphstyle=\color{red}]{../listings/letter_grades2.bash}
		}
		\end{block}
		\begin{block}{}
		{\scriptsize
		 \lstinputlisting[language=python, numbers=none]{../listings/grades_shell2_output_script.txt}
		}
		\end{block}
		\column{0.48\textwidth}
		\uncover<2->{
			\begin{block}{}
			{\scriptsize
			 \lstinputlisting[language=csh]{../listings/letter_grades.tcsh}
			}
			\end{block}
			\begin{block}{}
			{\scriptsize
			 \lstinputlisting[language=bash, numbers=none]{../listings/grades_tcsh_output_script.txt}
			}
			\end{block}
		}
	\end{columns}
\end{frame}

\begin{frame}
\frametitle{Overview}
	\begin{itemize}
		\item programming languages provide functions differently
		\item functions in the shell are a bit clunky (bash) or not available (tcsh), but workarounds possible
		\item think about how to parameterize your function
	\end{itemize}
\end{frame}
\begin{frame}
\frametitle{Syntax}
	\begin{itemize}
		\item some languages provide multiple return values:
			\begin{block}{}
				{\scriptsize
				 \lstinputlisting[language=python]{../listings/multi_return.py}
				}
			\end{block}
			\begin{block}{}
				{\scriptsize
				 \lstinputlisting[language=python, numbers=none]{../listings/multi_return.txt}
				}
			\end{block}
		\item<2-> could always return more complex type with multiple values (arrays, lists, \dots)
		
	\end{itemize}
\end{frame}

\begin{frame}
\frametitle{Examples - Python: multiple functions}
	\vspace{-0.25cm}
		\begin{block}{}
		{\scriptsize
		 \lstinputlisting[language=python, lastline=23]{../listings/grades2.py}
		}
		\end{block}
\end{frame}

\begin{frame}
\frametitle{Examples - Interactive Python}
		\begin{block}{import {\bf entire module}}
		{\scriptsize
		 \lstinputlisting[language=python, numbers=none]{../listings/grades2_output.txt}
		}
		\end{block}
		\uncover<2->{
			\begin{block}{import {\bf individual function{\color{red}s}} from a module}
			{\scriptsize
			 \lstinputlisting[language=python, numbers=none ]{../listings/grades2_output2.txt}
			}
			\end{block}
		}
\end{frame}


\begin{frame}
\frametitle{What is this \{'A':4.0\} Business?}
	\begin{itemize}
		\item python provides {\tt dictionary} datatype, associative arrays
		\item think "Lookup tables"
		\item you get to assign key value pairs where the key can be string, number, other datatype (list requires the key to be the index of the value)
	\end{itemize}

		\uncover<2->{
			\begin{block}{}
			{\tiny
			 \lstinputlisting[language=python, numbers=none ]{../listings/dict_example.txt}
			}
			\end{block}
		}
\end{frame}

\end{document}
